\documentclass{article}
% Comment the following line to NOT allow the usage of umlauts
\usepackage[utf8]{inputenc}
% Uncomment the following line to allow the usage of graphics (.png, .jpg)
%\usepackage{graphicx}

% Start the document
\begin{document}

% Create a new 1st level heading
\section{THE FUTURE OF HEALTH}

    Digital transformation, enabled by fundamentally interoperable data and open, secure platforms, will most certainly drive the future of health. Rather of responding to disease, health is more likely to centre around maintaining well-being. Cancer and diabetes may be able to join polio as diseases that have been eradicated in the next 20 years. Prevention and early diagnosis, we believe, will be crucial in the future of health. In rare circumstances, the development of sickness might be postponed or completely avoided. Because to advanced testing and instruments, most diagnosis (and treatment) may be done at home. The current health-care system in the United States is a jumble of disparate elements (health plans, hospital systems, pharmaceutical companies, medical device manufacturers).

    We anticipate that by 2040, the customer will be at the core of the health-care model. Interoperable, always-on data will encourage greater collaboration among industry stakeholders, and incumbents and new entrants will provide innovative service combinations (disruptors). Interventions and treatments will almost certainly be more precise, less complicated, less intrusive, and less expensive. Health will be defined holistically as a condition of mind, social, emotional, physical, and spiritual well-being. Consumers will not only have access to extensive health information, but they will also control their health data and play a vital part in determining health and well-being decisions.  

\section{WHAT IS FUTURE OF HEALTH}

    We are just around 20 years away from the health future we imagine, yet health in 2040 will be a world apart from what we have now. We may fairly expect digital transformation—enabled by fundamentally interoperable data, artificial intelligence (AI), and open, secure platforms—to drive much of this change, based on coming technologies. We believe that, unlike now, care will be structured around the customer rather than the institutions that currently drive our health-care system. Streams of health data, together with data from a range of other relevant sources, will merge by 2040 (and maybe much before) to generate a complex and highly personalised picture of each consumer's health. Wearable devices that measure our steps, sleep habits, and even heart rate have become an integral part of our lives in ways we could never have imagined only a few years ago. This is a pattern we believe will continue. For example, the next generation of sensors will transition us away from wearable gadgets and toward invisible, always-on sensors implanted in the objects that surround us.

    Many healthcare technology businesses are already incorporating always-on biosensors and software into data-generating, data-gathering, and data-sharing devices. Advanced cognitive technologies might be created to assess a huge number of variables and provide personalised health information to consumers. Precision well-being and real-time microinterventions can be enabled by the availability of data and personalised AI, allowing us to stay ahead of sickness and well ahead of catastrophic disease. Consumers will almost certainly demand that their health information be portable if they have this extremely comprehensive personal information about their own health. Consumers have been accustomed to revolutions in other industries, such as e-commerce and mobility.

    These customers will demand that health follow suit and become a seamless part of their lives, and they will vote with their feet and money. While we can't predict how the future will unfold with certainty, we can use current market signals—as well as forces of change in other industries—to begin to construct a picture of health's future. Fundamental transformations in invention occur in seven-year cycles in almost every business . Health is no exception. Three of these cycles will have passed by 2040, each one building on the last. We should go back three innovation cycles to see where exponential innovation has taken us to determine where health is heading.

\section{WHY DOES THE FUTURE OF HEALTH MATTER}

    Nothing is more vital to our well-being than our health. To varied degrees, we all engage with the health-care system, and we will continue to do so throughout our lives. Individuals, families, and companies, as well as municipal, state, and federal budgets, are all affected by the expense of health care. In 2017, the United States spent more than 3.5 trillion dollars on health care. At least one chronic ailment (such as heart disease, asthma, cancer, or diabetes) affects an estimated 133 million Americans, and the number of persons with a chronic illness has been gradually increasing for years.

      However, rather than focusing on treatment, the future of health will be centred on well-being and prevention. By 2040, we expect increasing health spending will be allocated to maintaining well-being and avoiding sickness, rather than diagnosing conditions and treating illness. Greater attention on well-being and early identification of health concerns will result in fewer and less severe diseases, lowering health-care costs and allowing the well-being dividend to be reinvested to benefit the entire population. Along with assisting people in improving their well-being, health care stakeholders will also seek to promote population health. Microinterventions that help individuals stay healthy will be driven by interoperable data sets.

    While certain health systems, health insurance, and life sciences corporations are beginning to change their attention to wellness, the overall system remains focused on sick care.

\section{IMPACTS OF THE FUTURE OF HEALTH}

    Existing stakeholders, new entrants, employers, and consumers will all be affected by the future of health. Many incumbents are naturally wary about bringing about change in a market that they now control. These firms may be well-positioned to lead from the front, given their solid footing in the current ecosystem and their ability to manage the regulatory environment. Companies centred on technology, such as Google, Amazon and Apple9, are beginning to disrupt and transform the market.

    Legacy stakeholders should think about whether they should disrupt themselves or isolate and defend their services in order to keep part of their market share. Some incumbent businesses may succumb to competition coming from outside the established industry boundaries, while others may be able to assist usher in the future of health. We predict new roles, functions, and actors to develop, mostly replacing the segmented industry divisions we currently have (such as health systems and physicians, health insurance, biopharmaceutical corporations, and medical device makers). Three major groups are expected to develop in the future of health (data and platforms, well-being and care delivery, and care enablement). 

    Hence, Future of Healthcare is really very Important.
% Uncomment the following two lines if you want to have a bibliography
%\bibliographystyle{alpha}
%\bibliography{document}

\end{document}